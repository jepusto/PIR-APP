\documentclass[11pt]{article}
\usepackage{amssymb}
\usepackage{amsmath}
\usepackage[margin=1in]{geometry}
\usepackage{graphicx}
\usepackage[natbibapa]{apacite}


\usepackage{subfig}
\usepackage{float}
\usepackage{multirow}
\usepackage{array}

\usepackage{titlesec}
\titleformat*{\section}{\large\bf}
\titleformat*{\subsection}{\normalsize\bf}

\newcommand{\E}{\text{E}}
\newcommand{\logit}{\text{logit}}
\newcommand{\cll}{\text{cll}}
\newcommand{\Var}{\text{Var}}
\newcommand{\Cov}{\text{Cov}}
\newcommand{\converge}[1]{\stackrel{\mathcal{{#1}}}{\rightarrow}}

\newcommand{\diag}{\text{diag}}
\newcommand{\tr}{\text{tr}}

\newcommand{\Info}{\mathcal{I}}
\newcommand{\bs}{\boldsymbol}
\newcommand{\bm}{\mathbf}

\begin{document}
Let $y_0,y_1,y_2,y_3,...,y_K$ be a sample from an equilibrium DTMC on an $(R+1)$-dimensional space with observed frequency counts $k_{qr} = \sum_{i=1}^K I(y_{i-1} = q, y_i = r)$ and transition probabilities $\pi_{qr} (\bs\theta) = \Pr(y_1 = r | y_0 = q)$ that are functions of a $p\times 1$ parameter vector $\bs\theta$. The equilibrium distribution of the DTMC has the properties that
\begin{equation}
\label{eq:balancing_normalizing}
 \Pr(y_0 = r) = \tilde{\pi}_r = \sum_{q=0}^R \tilde{\pi}_q \pi_{qr} \qquad \text{and} \qquad \sum_{r=0}^R \tilde{\pi}_r = 1.
\end{equation}
The log-likelihood is given by 
\begin{equation}
\label{eq:log_lik}
l(\bs\theta | y_0,...,y_K) = \sum_{q=0}^R I(y_0 = q) \log \tilde{\pi}_q + \sum_{q=0}^R \sum_{r=0}^R k_{qr} \log \pi_{qr}.
\end{equation}
The $i^{th}$ entry of the score function is therefore given by
\begin{equation}
\label{eq:score}
\frac{\partial l}{\partial \theta_i} = \sum_{q=0}^R \frac{I(y_0 = q)}{\tilde{\pi}_q} \frac{\partial \tilde{\pi}_q}{\partial \theta_i} +  \sum_{q=0}^R \sum_{r=0}^R \frac{k_{qr}}{\pi_{qr}} \frac{\partial \pi_{qr}}{\partial \theta_i}
\end{equation}
and the $(i,j)^{th}$ entry in the Hessian of the likelihood by
\begin{multline}
\label{eq:info_observed}
\frac{\partial^2 l}{\partial \theta_i \partial \theta_j} = \sum_{q=0}^R \left(\frac{I(y_0 = q)}{\tilde{\pi}_q} \frac{\partial^2 \tilde{\pi}_q}{\partial \theta_i \partial \theta_j} - \frac{I(y_0 = q)}{\tilde{\pi}_q^2} \frac{\partial \tilde{\pi}_q}{\partial \theta_i} \frac{\partial \tilde{\pi}_q}{\partial \theta_j}\right) \\
 +  \sum_{q=0}^R \sum_{r=0}^R \left(\frac{k_{qr}}{\pi_{qr}} \frac{\partial^2 \pi_{qr}}{\partial \theta_i \partial \theta_j} - \frac{k_{qr}}{\pi_{qr}^2} \frac{\partial \pi_{qr}}{\partial \theta_i}  \frac{\partial \pi_{qr}}{\partial \theta_j} \right).
\end{multline}
Because the DTMC is in equilibrium, $E(I(y_0 = q)) = \Pr(y_0 = q) = \tilde{\pi}_q$ and \[
E(k_{qr}) = K \Pr(y_0 = q, y_1 = r) = K \Pr(y_0 = q) \Pr(y_1 = r | y_0 = q) = K \tilde{\pi}_q \pi_{qr}. \]
Furthermore, from the property of transition matrices that $\sum_{r=1}^R \pi_{qr} = 1$, it follows that \[
\sum_{r=0}^R \frac{\partial \pi_{qr}}{\partial \theta_i} = 0 \qquad \text{and} \qquad \sum_{r=0}^R \frac{\partial^2 \pi_{qr}}{\partial \theta_i \partial \theta_j} = 0; \]
similarly, \[
\sum_{r=0}^R \frac{\partial \tilde{\pi}_r}{\partial \theta_i} = 0 \qquad \text{and} \qquad \sum_{r=0}^R \frac{\partial^2 \tilde{\pi}_r}{\partial \theta_i \partial \theta_j} = 0. \]
The expected information matrix therefore has $(i,j)^{th}$ entry
\begin{equation}
\label{eq:info_expected_full}
-\E\left(\frac{\partial^2 l}{\partial \theta_i \partial \theta_j} \right) =  \sum_{q=0}^R \frac{1}{\tilde{\pi}_q} \frac{\partial \tilde{\pi}_q}{\partial \theta_i} \frac{\partial \tilde{\pi}_q}{\partial \theta_j} +  K \sum_{q=0}^R \sum_{r=0}^R \frac{\tilde{\pi}_q}{\pi_{qr}} \frac{\partial \pi_{qr}}{\partial \theta_i}  \frac{\partial \pi_{qr}}{\partial \theta_j}.
\end{equation}
For purposes of comparing the efficiency of different recording procedures, the first term in (\ref{eq:info_expected_full}) can be ignored as $K$ grows large. Thus I use 
\begin{equation}
\label{eq:info_expected}
\Info^E_{ij} = K \sum_{q=0}^R \sum_{r=0}^R \frac{\tilde{\pi}_q}{\pi_{qr}} \frac{\partial \pi_{qr}}{\partial \theta_i}  \frac{\partial \pi_{qr}}{\partial \theta_j}.
\end{equation}

\section{Momentary time sampling}

The transition matrix corresponding to momentary time sampling with intervals of length $L$ is given by 
\begin{equation}
\bs\Pi^M = \left[\begin{array}{cc} 1 - p_0(L) & p_0(L) \\ 1 - p_1(L) & p_1(L) \end{array}\right],
\end{equation}
where $p_0(t) = \phi \left(1 - e^{- \rho l} \right)$ and $p_1(t) = (1 - \phi) e^{- \rho t} + \phi$. The equilibrium distribution of this DTMC is given by $\tilde{\pi}_0 = 1 - \phi$, $\tilde{\pi}_1 = \phi$. The first derivatives of the transition probabilities with respect to $\phi$ and $\rho$ are given by \begin{align*}
\frac{\partial \bs\Pi^M}{\partial \phi} &= \left[\begin{array}{cc} - (1 - e^{- \rho L}) & (1 - e^{- \rho L}) \\  - (1 - e^{- \rho L}) & (1 - e^{- \rho L}) \end{array}\right], \\
\frac{\partial \bs\Pi^M}{\partial \rho} &= \left[\begin{array}{cc} - \phi L  e^{- \rho L} & \phi L e^{- \rho L} \\  (1 - \phi) L e^{- \rho L} & -(1 - \phi) L e^{- \rho L} \end{array}\right].
\end{align*}
The expected information matrix corresponding to the momentary time sampling procedure can then be evaluated directly from (\ref{eq:info_expected}). 

\section{Partial interval recording}

The transition probabilities corresponding to partial interval recording with intervals of length $L$ are given by 
\begin{equation}
\pi_{j,j+1} = 1 - e^{-\phi \rho L} \left[1 - \stackrel{(j)}{\circ}f \left(0 \right) \right], \qquad \pi_{j,0} = 1 - \pi_{j,j+1}
\end{equation}
for $j = 0,1,2,3,...$, where \[
f(q) = \frac{\phi - \left(\phi - q \right) e^{-\rho L}}{1 - (1 - q) e^{-\phi \rho L}} \]
and $\stackrel{(j)}{\circ}f$ denotes $j$-fold recursion of $f$: \[
\stackrel{(0)}{\circ}f(q) = q, \qquad \stackrel{(j)}{\circ}f(q) = f\left(\stackrel{(j-1)}{\circ}f(q) \right). \]
The equilibrium distribution of this DTMC is given by \[
\tilde{\pi}_0 = 1 - E(U_k) = (1 - \phi) e^{-\phi \rho L}, \]
and from (\ref{eq:balancing_normalizing}), \[
\tilde{\pi}_j = \tilde{\pi}_{j-1} \pi_{j-1,j} = \tilde{\pi}_0 \prod_{i=1}^j \pi_{i-1,i} \]
for $j=1,2,3,...$. The derivatives of $\pi_{j,j+1}$ are given by \begin{align*}
\frac{\partial \pi_{j,j+1}}{\partial \phi} &= \rho L \left(1 - \pi_{j,j+1}\right) +  e^{-\phi \rho L} \left(\stackrel{(j)}{\circ}g_\phi(0,0) \right), \\
\frac{\partial \pi_{j,j+1}}{\partial \rho} &= \phi L \left(1 - \pi_{j,j+1}\right) +  e^{-\phi \rho L} \left(\stackrel{(j)}{\circ}g_\rho(0,0) \right), 
\end{align*}
for $j = 0,1,2,...$. Here, $g_\phi$ and $g_\rho$ are two-dimensional functions defined by $g_\theta(q,r) = \left(f(q), f_\theta'(q,r) \right)$ with \begin{align*}
f'_{\phi}(q,r) &= \frac{\partial f(q)}{\partial \phi} = \frac{1 - e^{-\rho L}\left(1 - r \right) - f(q) e^{-\phi \rho L} \left[\rho L(1 - q) + r \right]}{1 - (1 - q) e^{-\phi \rho L}} \\
f'_{\rho}(q,r) &= \frac{\partial f(q)}{\partial \rho} = \frac{e^{-\rho L}\left[L(\phi - q) + r\right] - f(q) e^{-\phi \rho L} \left[\phi L(1 - q) + r \right]}{1 - (1 - q) e^{-\phi \rho L}},
\end{align*}
where I write $r$ for $\frac{\partial q}{\partial \theta}$. 

The expected information matrix for this DTMC can then be evaluated as
\begin{equation}
\Info_E^P = K \sum_{q=0}^\infty \frac{\tilde{\pi}_q}{\pi_{q,q+1}(1 - \pi_{q,q+1})} \frac{\partial \pi_{q,q+1}}{\partial \theta_i}  \frac{\partial \pi_{q,q+1}}{\partial \theta_j}.
\end{equation}

\section{Augmented interval recording}

The transition matrix corresponding to augmented interval recording with intervals of length $L$ is given by 
\begin{equation}
\bs\Pi^A = \left[\begin{array}{cccc} e^{-\phi \rho L} & 1 - e^{-\phi \rho L} - p_0(L) & p_0(L) & 0 \\ 0 & \frac{1 - \phi}{\phi}p_0(L) & 1 - e^{-(1 - \phi) \rho L} - \frac{1 - \phi}{\phi}p_0(L)& e^{-(1 - \phi) \rho L}  \end{array}\right].
\end{equation}
The equilibrium distribution of this DTMC is given by $\tilde{\pi}_0 = 1 - \phi$, $\tilde{\pi}_1 = \phi$. The first derivatives of the transition probabilities with respect to $\phi$ and $\rho$ are given by \begin{align*}
\frac{\partial \bs\Pi^A}{\partial \phi} &= \left[\begin{array}{cccc}
-\rho L e^{-\phi \rho L} & \rho L e^{-\phi \rho L} - (1 - e^{-\rho L}) & (1 - e^{-\rho L}) & 0 \\
0 & - (1 - e^{-\rho L})  &  1 - e^{-\rho L}  - \rho L e^{-(1 - \phi) \rho L} & \rho L e^{-(1 - \phi) \rho L}
\end{array}\right], \\
\frac{\partial \bs\Pi^A}{\partial \rho} &= \left[\begin{array}{cccc} 
-\phi L e^{-\phi \rho L} &\phi L \left( e^{-\phi \rho L} - e^{-\rho L}\right) & \phi L e^{-\rho L} & 0 \\ 
0 & (1 - \phi) L e^{-\rho L} & (1 - \phi) L \left(e^{-(1-\phi) \rho L} - e^{-\rho L}\right) & - (1 - \phi) L e^{- (1 - \phi) \rho L} 
\end{array}\right].
\end{align*}
The expected information matrix corresponding to the momentary time sampling procedure can then be evaluated directly from (\ref{eq:info_expected}). 



\end{document}